\documentclass{article}
\usepackage{a4}
\bibliographystyle{unsrt}

\title{Non-proprietary name diversity for antibody-based drugs}
\author{Andrew C.R.\ Martin,\\Institute of Structural and Molecular
  Biology,\\Division of Biosciences,\\University College
  London,\\Darwin Building, Gower Street,\\London WC1E 6BT.}
\begin{document}
\maketitle

\begin{abstract}
Antibody-based drugs have become an extremely important class of
pharmaceuticals. Approximately one quarter of drugs assigned an
International Nonproprietary Name (INN) by the World Health
Organization (WHO) are antibodies.  The restrictions of the standard
INN naming scheme and the enormous number of names being assigned,
means that the available namespace is becoming densely populated.
Names that are too similar can lead to confusion and prescribing
errors. Here we investigate the diversity of the antibody name space
and show that the diversity has decreased over time. We provide a tool
that allows a potential new antibody name to be compared with
currently used names to help with selection of diverse names.
\end{abstract}

\section{Introduction}
Antibodies offer a number of advantages as drugs. They can act in the
same way that conventional drugs do (generally by binding to enzymes
or receptors and switching them on or off), but their large
interacting surface allows them to bind to targets (such as hormones,
growth factors and toxins) that small molecule drugs cannot bind
effectively since they do not have pockets into which small molecules
must bind. Secondly, antibodies can be used to trigger the iummune
system to kill cells or viruses. Thirdly, they can act to target other
drugs to the correct location in so-called `antibody-drug conjugates'
(ADCs). In addition, antibody-based drugs are being engineered to
allow cross-linking --- for example bringing cells of the immune
system into close proximity with cancer cells. Over the last 10--15
years, antibody-based drugs have become an incredibly important class
of pharmaceuticals with the majority of pharmaceutical companies and
many small biotechnology companies having antibody-based drugs in
their development pipelines\cite{xxx}. Approximately half of the top
ten selling blockbuster drugs are antibodiesc\cite{xxx} with
approximately one third of drugs in development being
antibodies\cite{xxx}.

The World Health Organization (WHO) assigns International
Nonproprietary Names (INN) to drugs. An INN allows a substance to be
identified uniquely and, when the substance comes off patent allows
other manufacturers to identify their products. The healthcare system
in many countries requires or encourages prescribing using
nonproprietary names since these are often considerably more cost
effective that original branded products. INNs are assigned to drugs
before they are licenced, typically during phase~II clinical
trials. The popularity of antibody-based drugs means that
approximately one quarter of all drugs that are being assigned an INN
are now antbodies.

The WHO uses a standard stem (\emph{-mab}) for almost all
antibody-based drugs, the exception being a small number of fusions of
antibodies with other biological drugs where the antibody is used only
to target the active substance to the correct site. Such molecules
have a \emph{-fusp} (`fusion protein') stem with the letter `\emph{a}'
appearing before the \emph{-fusp} to indicate that part of the fusion
is an antibody. Until recently, an infix before the \emph{-mab} was
used to name the source of the antibody. For example, \emph{-zu-} was
used to indicate the antibody had been humanized, \emph{-o-} for a
mouse antibody, \emph{-xi-} for a chimeric and \emph{-u-} for
human\cite{xxx}. However, the complexity of methods by which
antibodies are produced and some controversies in the ways in which
such classifications were assigned meant that these infixes were
removed from May 2017\cite{xxx}.


Another reason for 

and sub-stems in naming antibodies, but the
enormous number of names being assigned means that the available
namespace is becoming densely populated. Names that are too similar
can lead to confusion and prescribing errors. Here we investigate the
diversity of the antibody name space and show that the diversity has
decreased over time. We provide a tool that allows a potential new
antibody name to be compared with currently used names to help with
selection of diverse names.

\section{Materials and Methods}

\section{Results}

\section{Discussion}

\bibliography{abbrev,checkname}
\end{document}


  

\documentclass{article}
\usepackage{a4}
\usepackage{url}
\usepackage{epsfig}
\bibliographystyle{unsrt}

\newcommand{\eg}[1]{`\emph{#1}'}

\title{Non-proprietary name diversity for antibody-based drugs}
\author{Andrew C.R.\ Martin,\\Institute of Structural and Molecular
  Biology,\\Division of Biosciences,\\University College
  London,\\Darwin Building, Gower Street,\\London WC1E 6BT.}
\begin{document}
\maketitle

\begin{abstract}
Antibody-based drugs have become an extremely important class of
pharmaceuticals. Approximately one quarter of drugs assigned an
International Nonproprietary Name (INN) by the World Health
Organization (WHO) are antibodies.  The restrictions of the standard
INN naming scheme and the enormous number of names being assigned,
means that the available namespace is becoming densely populated.
Names that are too similar can lead to confusion and consequently to
prescribing errors. Here we investigate the diversity of the antibody
name space and show that the diversity has decreased over time. We
provide a tool that allows a potential new antibody name to be
compared with currently used names to help with selection of diverse
names.
\end{abstract}

\section{Introduction}
Antibodies offer a number of advantages as drugs. They can act in the
same way that conventional drugs do (generally by binding to enzymes
or receptors and switching them on or off), but their large
interacting surface allows them to bind to targets (such as hormones,
cytokines, growth factors and toxins) that small molecule drugs cannot
bind effectively since these targets do not have pockets into which small
molecules must bind. Secondly, antibodies can be used to trigger the
iummune system to kill cells or viruses. Thirdly, they can act to
target other drugs to the correct location in so-called `antibody-drug
conjugates' (ADCs). In addition, antibody-based drugs are being
engineered to allow cross-linking --- for example bringing cells of
the immune system into close proximity with cancer cells.

Over the last 10--15 years, antibody-based drugs have become an
incredibly important class of pharmaceuticals with the majority of
pharmaceutical companies and many small biotechnology companies having
antibody-based drugs in their development
pipelines\cite{xxx}. Approximately half of the top ten selling
blockbuster drugs are antibodies\cite{xxx} with approximately one
third of drugs in development being antibodies\cite{xxx}.

The World Health Organization (WHO) assigns International
Non-proprietary Names (INNs) to drugs. An INN allows a substance to be
identified uniquely and, when the patent for the substance expires,
the INN allows other manufacturers to identify their products. The
healthcare system in many countries requires or encourages prescribing
using non-proprietary names since generic products are often
considerably more cost effective that original branded products. INNs
are assigned to drugs before they are licenced, typically during
phase~II clinical trials. The popularity of antibody-based drugs means
that approximately one quarter of all drugs that are being assigned an
INN are now antibodies with up to 100 names being assigned per year.

The WHO uses a standard stem (\eg{-mab}) for almost all
antibody-based drugs, the exception being a small number of fusions of
antibodies with other biologic drugs where the antibody is used only
to target the active substance to the correct site. These molecules
have a \eg{-fusp} (`fusion protein') stem with the letter \eg{-a-}
appearing before the \eg{-fusp} to indicate that part of the fusion
is an antibody. Until recently, an infix before the \eg{-mab} was
used to name the source of the antibody. For example, \eg{-zu-} was
used to indicate the antibody had been humanized, \eg{-o-} for a
mouse antibody, \eg{-xi-} for a chimeric and \eg{-u-} for
human\cite{xxx} leading to antibody names frequently ending in
\eg{-zumab}, \eg{-umab}, or \eg{-ximab}. However, it became
clear that capturing the complexity of methods by which antibodies are
now produced in a single syllable was becoming impossible. This,
together with some controversies in the ways in which such
classifications were assigned, meant that these infixes were removed
from antibody names assigned from May 2017\cite{xxx}. Another infix is
also present which indicates the `target' for the antibody, although in
some cases this is more indicative of an initial indication for the
drug. The target substem includes \eg{-li-} for immunomodulatory,
\eg{-ci-} for circulatory system and \eg{-ta-} (previously
\eg{-tu-}) for tumours.

Thus, while trying to keep the name below five syllables, having a
fixed stem (\eg{-mab}) and two infixes, the available name space was
severely limited and this was another reason for removing the source
infix. Even so, with the majority of antibodies targeting the immune
system or tumours, the vast majority of antibodies now being named end
in \eg{-limab} or \eg{-tamab} meaning that the diversity of available
names is still limited. Other restrictions on names include the need
to avoid relevant trademarks, or names that could be associated with a
company as well as avoiding names that could be perceived to offer a
commercial advantage or disadvantage (perhaps because they are
considered rude or offensive). Since these names are used
internationally, such considerations must be made across a broad range
of languages and not just English.

The enormous number of names being assigned means that the available
namespace is becoming densely populated. Names that are too similar
can lead to confusion and hence to prescribing errors. Here we
investigate the diversity of the antibody name space using tools from
computational linguistics and show that the diversity has decreased
over time. We provide a tool that allows a potential new antibody name
to be compared with currently used names to help with selection of
diverse names.

\section{Materials and Methods}
Assigned antibody names were collected from the Mednet portal to the
INN data (\url{mednet-communities.net/inn}) covering all data from
proposed lists up to 118 and recommended lists up to 78 both released
in Autumn 2017.

Comparison of names is essentially a problem of performing an optimal
alignment of two strings and is exactly the same problem faced in
sequence alignment. Standard dynamic programming algorithms for
local\cite{sw:alignment} or global\cite{nw:alignment} alignment are
employed to align two strings using appropriate scoring matrices. In
the case of protein sequence alignment, those scoring matrices are
generally either Dayhoff type matrices\cite{dayhoff:mdm78} or BLOSUM
matrices\cite{henikoff:blosum}. In the case of linguistics, different
matrices may be chosen depending on the sorts of comparisons required:
one may wish to examine the similarity of words when written or when
spoken.

When examining the sounds of words, general computational linguistics
usually applies a pre-processing step known as grapheme-to-phoneme
(G2P) translation\cite{hixon:xxx}.  Phonemes are the perceptually
distinct units of sound that make up words. These are often written
using the International Phonetic Alphabet (IPA), a standardized
representation of the sounds of spoken
language\cite{mcmahon:phoneticnotation} familiar to anyone who has
looked in a dictionary that contains pronunciation guidance. For
example, an obvious requirement would be to replace the letters
\eg{ph} and \eg{f} with the same phonetic character (`{\bfseries\it
  f}') and this G2P translation is an essential step in both automatic
speech recognition and text to speech applications.  However, for the
problem of comparing antibody names, this step is somewhat
superfluous. There is no `correct' pronunciation for these words and
dipthongs (either diagraphs representing a single sound such as in the
word \eg{feat} or compound vowel character ligatures such as \eg{\ae})
as well as double consonants, double vowels and constructions such as
\eg{ph} are not present. Thus for our purposes, the names of
antibodies can be compared directly with no G2P translation.

Kondrak introduced a phonetic similarity matrix based around the
standard 26 letters of the
alphabet\cite{kondrak:paper,kondrak:thesis}. This methodology is
encoded in his ALINE program.  He classifies the base sound made by
each letter according to 12 features: Syllabic, Place, Voice, Nasal,
Lateral, Aspirated, High, Back, Manner, Retroflex, Long and
Round. Some of these are binary while others are multi-valued (see
Tables~4.28, 4.30 and~4.31 of \cite{kondrak:thesis}). Importantly, he
also provides a set of saliancies (or weights) for the features (see
Table~4.27 of \cite{kondrak:thesis}) and weights that differ depending
on whether the comparison does or does not include a consonant. He
provides a set of equations (Table~4.26 of \cite{kondrak:thesis}) that
can be applied to these features to calculate the similarity score for
any pair of letters and we have implemented these rules in the program
\verb|MakeKondrakMatrix| written in C.  Corrected features for the
letter \eg{y} (or more strictly the non-syllabic vocoid or `glide'
represented by the character \eg{y}) were kindly provided by Grzegorz
Kondrak.

As well as spoken similarity, the possibility of confusing antibody
names when written is also important, particularly in hand-written
notes or in prescriptions. Simpson\cite{simpson:xxx} provides a letter
confusion (or similarity matrix) which was downloaded as an Excel file
from the paper's supplementary material.

\begin{table}
\begin{center}
\begin{tabular}{llllllllllllll}\hline
          & a & b & c & d & e & f & g & h & i & j & k & l & m \\ \hline
Ascender  & 0 & 2 & 0 & 2 & 0 & 2 & 0 & 2 & 1 & 1 & 2 & 2 & 0 \\
Descender & 0 & 0 & 0 & 0 & 0 & 1 & 2 & 0 & 0 & 2 & 0 & 0 & 0 \\ \hline
          &   &   &   &   &   &   &   &   &   &   &   &   &   \\
          & n & o & p & q & r & s & t & u & v & w & x & y & z \\ \hline
Ascender  & 0 & 0 & 0 & 0 & 0 & 0 & 1 & 0 & 0 & 0 & 0 & 0 & 0 \\
Descender & 0 & 0 & 2 & 2 & 0 & 0 & 0 & 0 & 0 & 0 & 0 & 2 & 0 \\ \hline
\end{tabular}
\end{center}
  \caption{\label{tab:boumascores} Scores assigned for ascenders and
    descenders of each of the letters. All descenders were given a
    score of two with the exception of the letter \eg{f} whose score
    of one represents the fact that it often appears in printed text
    without a descender.}
\end{table}

Bouma\cite{xxx} introduced the importance of the shapes of words in
reading text. The `Bouma' of a word written in all-capitals is always
a rectangle, but lower-case words have ascenders and descenders
protruding from the basic rectangle. The importance of word shape in
learning to read and in word recognition has been widely
discussed\cite{xxx}.  A program \verb|MakeBoumaMatrix| was written in
Perl to generate a Bouma similarity matrix based on the equation:
\begin{equation}
  B_{x,y} = M - (|A_x - A_y| + |D_x - D_y|)
\end{equation}
\begin{equation}
  M={\mathrm max}\{D_i, A_i\}_{(i\in \{a\ldots z\})}
\end{equation}
where $A_x$ is the ascender score for character $x$, $D_x$ is the
descender score for character $x$, $M$ is the maximal ascender or
descender score, and $B_{x,y}$ is the Bouma similarity for two letters
$x$ and $y$. Values for the ascender and descender scores are shown in
Table~\ref{tab:boumascores}.

The Needleman and Wunsch dynamic programming sequence alignment code
available in the BiopLib\cite{porter:bioplib} library was used in the
program \verb|abnamecheck| implemented in C. \verb|abnamecheck| allows
the analysis of all current names and allows a new name to be compared
against existing names.

Non-affine gap penalties were used (i.e.\ the same penalty for
introducing and extending a gap). Initially gap penalties were set to
equal the maximum score seen in the respective scoring matrices and
then reduced until acceptable alignments were observed. For the
Kondrak phonetic alignments and Simpson letter similarity alignments,
it was required that the stem (\eg{-mab}) was correctly aligned and
for the Bouma alignments, it was required that the final letter \eg{b}
was aligned for at least 99.9\% of alignments.

From the analysis of existing names, the mean and standard deviation
of the similarities for each of these three scoring schemes was
calculated and a threshold of $\bar{x} + 3\sigma$ was used as a
default for identifying names that are significantly similar to names
already in the dataset.

The numbers of approved antibodies each year were obtained from the
`Approved Antbodies List' maintained by Janice Reichert and available
from the members area of the web site of the Antibody Society.



\begin{figure}
  \noindent a)
  
  \psfig{file=figures/namesperyear_cummulative.eps, width=\linewidth}

  \noindent b)

  !!!TODO!!!
  
  \caption{\label{fig:namesandapproved}Cummulative number of antibodies
  named each year since 1981 when the \eg{-mab} stem was introduced and
  numbers of antibodies given marketing approval in Europe or the
  USA.}
\end{figure}



\begin{figure}
\begin{center}
  \psfig{file=figures/distribution.eps, width=0.7\linewidth}
\end{center}
 \caption{\label{fig:distrib} Distribution of antibody name similarity
   scores using 5\% bins. Similarity is scored using Kondrak phonetic
   similarity, Simpson letter similarity and the Bouma word-shape
   similarity.}
\end{figure}



\begin{figure}
  \noindent a)

\begin{center}
  \psfig{file=figures/mean_byyear.eps, width=0.7\linewidth}
\end{center}

  \noindent b)

\begin{center}
  \psfig{file=figures/sd_byyear.eps, width=0.7\linewidth}
\end{center}

  \noindent c)

\begin{center}
  \psfig{file=figures/meanplus3sd_byyear.eps, width=0.7\linewidth}
\end{center}

  
  \caption{\label{fig:results} Antibody name similariry
    scores plotted against year. a)~the mean score; b)~the standard
    deviation; c)~a threshold defined as $\bar{x}+3\sigma$}
\end{figure}



\section{Results}
The first antibody to be given an INN was XXXX before the \eg{-mab}
stem was introduced; consequently this example was excluded from the
analysis. Antibody names extracted from the proposed and recommended
lists are shown in Supplementary Table~1 together with their list
numbers and the year in which their names appeared in the recommended
list or in the propsed list if they did not appear in a recommended
list. The number of antibodies named and the number approved each year
are shown in Figure~\ref{fig:namesandapproved}.

The distribution of similarity scores for the full data set is shown in
Figure~\ref{fig:distrib}. 

The mean similarity scores resulting from pairwise comparison of all
antibody names, were calculated for each year in a cummulative fashion
(for example the scores for 2010 were calculated for antibody names
publsihed in 2010 and in all previous years). Results are shown in
Figure~\ref{fig:results}. 

The results show XXX.

\section{Discussion}
Over the last 10--15 years, antibody-based drugs have become a major
class of novel therapeutics with some 60--100 such drugs are being
named each year. Only a small percentage of these are successful in
gaining marketing authorizations in Europe or the USA so many of these
names are `wasted'. The restrictions imposed by having a standard
naming scheme means that the name space available for assigning
non-proprietary names to these drugs is very restricted.

We have shown that the name space has become more saturated over time,
but the changes made in 2017 (removing the source infix from the
naming scheme) has increased the diversity of names. Nonetheless, we
expect the rate at which antibody-based drugs are developed to
continue increasing. Extrapolating from
Figures~\ref{fig:namesandapproved} and~\ref{fig:results} we expect it
to be impossible to find names that are unlikely to be confused by
XXXX.

We have provided a tool that may help drug companies in suggesting
names which can be downloaded from \url{www.bioinf.org.uk/software/}
or from \url{www.github.com/AndrewCRMartin/XXXX}. The tool may also be
accessed for use over the web at \url{www.bioinf.org.uk/abs/names/}


\bibliography{abbrev,checkname}
\end{document}


%% Supplementary Table S1 - antibody names, INN lists and years in
%% which they appeared



  

\documentclass{article}
\usepackage{a4}
\usepackage{url}
\usepackage{epsfig}
\bibliographystyle{unsrt}
\emergencystretch 1in

\newcommand{\eg}[1]{\mbox{`\emph{#1}'}}
\usepackage{color}
\newcommand{\note}[1]{\mbox{\color{red}[*** #1 ***]}}

\title{Non-proprietary name diversity for antibody-based drugs}
\author{Andrew C.R.\ Martin,\\Institute of Structural and Molecular
  Biology,\\Division of Biosciences,\\University College
  London,\\Darwin Building, Gower Street,\\London WC1E 6BT.}
\begin{document}
\maketitle

\begin{abstract}
Antibody-based drugs have become an extremely important class of
pharmaceuticals. Approximately one quarter of drugs assigned an
International Nonproprietary Name (INN) by the World Health
Organization (WHO) are antibodies.  The restrictions of the standard
INN naming scheme and the enormous number of names being assigned,
means that the available namespace is becoming densely populated.
Names that are too similar can lead to confusion and consequently to
prescribing errors. Here we investigate the diversity of the antibody
name space and show that the diversity has decreased over time. We
provide a tool that allows a potential new antibody name to be
compared with currently used names to help with selection of diverse
names.
\end{abstract}

\section{Introduction}
Antibodies offer a number of advantages as drugs. They can act in the
same way that conventional drugs do (generally by binding to enzymes
or receptors and switching them on or off), but their large
interacting surface allows them to bind to targets (such as hormones,
cytokines, growth factors and toxins) that small molecule drugs cannot
bind effectively since these targets do not have pockets into which
small molecules must bind. Secondly, antibodies can be used to trigger
the immune system to kill cells or viruses. Thirdly, they can act to
target other drugs to the correct location in so-called `antibody-drug
conjugates' (ADCs). In addition, antibody-based drugs have been
engineered to produce bi-specific and multi-specific constructs which
allow cross-linking --- for example bringing cells of the immune
system into close proximity with cancer cells.

Over the last 10--15 years, antibody-based drugs have grown to become
the dominant product class within the biopharmaceutical
market\cite{ecker:abmarket} with many pharmaceutical companies and
small biotechnology companies having antibody-based drugs in their
development pipelines. It has been estimated that approximately one
third of drugs in development being antibodies\cite{reichert:trends}
and half of the top-ten-selling blockbuster drugs are
antibodies\cite{igea:top10:2018} including the world's top selling
prescription drug in 2017, with global revenues of \$18.43bn.

The World Health Organization (WHO) assigns International
Nonproprietary Names (INNs) to drugs. An INN allows a substance to be
identified uniquely and, when the patent for the substance expires,
the INN allows other manufacturers to identify their products allowing
generic prescribing. The healthcare system in many countries requires
or encourages prescribing using nonproprietary names since generic
products are often considerably more cost effective that original
branded products. INNs are assigned to drugs before they are licenced,
typically during phase~II clinical trials. The popularity of
antibody-based drugs means that approximately one quarter of all drugs
that are being assigned an INN are now antibodies with up to 100 names
being assigned per year, but less than 10\% of antibodies that have
assigned names are approved for use.

The WHO uses a standard stem (\eg{-mab}) for almost all antibody-based
drugs, the exception being a small number of fusions of antibodies
with other biologic drugs where the antibody is used only to target an
active biologic substance to the correct site. These molecules have a
\eg{-fusp} (`fusion protein') stem with the letter \eg{-a-} appearing
before the \eg{-fusp} to indicate that part of the fusion is an
antibody. In addition, fusions where only the Fc fragment of an
antibody is used in order to increase the drug's half life are instead
given a prefix of \eg{ef-}. Until recently, an infix before the
\eg{-mab} was used to name the source of the antibody. For example,
\eg{-zu-} was used to indicate the antibody had been humanized,
\eg{-o-} for a mouse antibody, \eg{-xi-} for a chimeric and \eg{-u-}
for human\cite{inn:stembook} leading to antibody names frequently
ending in \eg{-zumab}, \eg{-umab}, or \eg{-ximab}. However, it became
clear that capturing the complexity of methods by which antibodies are
now produced in a single syllable was becoming impossible. This,
together with some controversies in the ways in which such
classifications were assigned, meant that these infixes were removed
from antibody names assigned from May
2017\cite{inn:2017scheme,parren:names2017}. Another infix is also
present which indicates the `target' for the antibody, although in
some cases this is more indicative of an initial indication for the
drug. The target substem includes \eg{-li-} for immunomodulatory,
\eg{-ci-} for circulatory system and \eg{-ta-} (previously \eg{-tu-})
for tumours.

Thus, while trying to keep the name below five syllables, having a
fixed stem (\eg{-mab}) and two infixes, the available name space was
severely limited and this was another reason for removing the source
infix. Even so, with the majority of antibodies targeting the immune
system or tumours, the vast majority of antibodies now being named end
in \eg{-limab} or \eg{-tamab} meaning that the diversity of available
names is still limited. Other restrictions on names include the need
to avoid relevant trademarks, or names that could be associated with a
company, as well as avoiding names that could be perceived to offer a
commercial advantage or disadvantage (perhaps because they are
considered rude or offensive). Since these names are used
internationally, such considerations must be made across a broad range
of languages and not just English.


The enormous number of names being assigned means that the available
namespace is becoming densely populated. Names that are too similar
can lead to confusion and hence to prescribing errors.  The US FDA
provides a tool, POCA (Phonetic and Orthographic Computer Analysis,
\url{https://www.fda.gov/drugs/resourcesforyou/industry/ucm400127.htm})
which enables the comparison of potential new names with existing
names, but it is a large and complex program requiring the Microsoft
.Net Framework 4.0 and the Oracle 12c Database.  Here we develop a
simple open source portable command line tool using methods from
computational linguistics to investigate the diversity of the antibody
name space. We show that, surprisingly, the mean similarity between
names has decreased over the last 10 years, but that the standard
deviations has also decreased.  The tool we have developed allows a
potential new antibody name to be compared with currently used names
to help with selection of diverse names.

\section{Materials and Methods}
Assigned antibody names were collected from the Mednet portal to the
INN data (\url{mednet-communities.net/inn}) covering all data from
proposed lists up to 118 and recommended lists up to 78 both released
in Autumn 2017.

Comparison of names is essentially a problem of performing an optimal
alignment of two strings and is exactly the same problem faced in
sequence alignment. Standard dynamic programming algorithms for
local\cite{sw:alignment} or global\cite{nw:alignment} alignment are
employed to align two strings using appropriate scoring matrices. In
the case of protein sequence alignment, those scoring matrices are
generally either Dayhoff type matrices\cite{dayhoff:mdm78} or BLOSUM
matrices\cite{henikoff:blosum}. In the case of linguistics, different
matrices may be chosen depending on the sorts of comparisons required:
one may wish to examine the similarity of words when written or when
spoken.

When examining the sounds of words, general computational linguistics
usually applies a pre-processing step known as grapheme-to-phoneme
(G2P) translation\cite{hixon:phonemicsimilarity}.  Phonemes are the perceptually
distinct units of sound that make up words. These are often written
using the International Phonetic Alphabet (IPA), a standardized
representation of the sounds of spoken
language\cite{mcmahon:phoneticnotation} familiar to anyone who has
looked in a dictionary that contains pronunciation guidance. For
example, an obvious requirement would be to replace the letters
\eg{ph} and \eg{f} with the same phonetic character (`{\bfseries\it
  f}') and this G2P translation is an essential step in both automatic
speech recognition and text to speech applications.  However, for the
problem of comparing antibody names, this step is somewhat
superfluous. There is no `correct' pronunciation for these words and
dipthongs (either diagraphs representing a single sound such as in the
word \eg{feat} or compound vowel character ligatures such as \eg{\ae})
as well as double consonants, double vowels and constructions such as
\eg{ph} are not present. Thus for our purposes, the names of
antibodies can be compared directly with no G2P translation.

Kondrak introduced a phonetic similarity matrix based around the
standard 26 letters of the
alphabet\cite{kondrak:paper,kondrak:thesis}. This methodology is
encoded in his ALINE program.  He classifies the base sound made by
each letter according to 12 features: Syllabic, Place, Voice, Nasal,
Lateral, Aspirated, High, Back, Manner, Retroflex, Long and
Round. Some of these are binary while others are multi-valued (see
Tables~4.28, 4.30 and~4.31 of Kondrak's PhD
thesis\cite{kondrak:thesis}).  Importantly, he also provides a set of
saliancies (or weights) for the features (see Table~4.27 of
\cite{kondrak:thesis}) and weights that differ depending on whether
the comparison does or does not include a consonant. He provides a set
of equations (Table~4.26 of \cite{kondrak:thesis}) that can be applied
to these features to calculate the similarity score for any pair of
letters and we have implemented these rules in the program
\verb|MakeKondrakMatrix| written in C.  Corrected features for the
letter \eg{y} (or more strictly the non-syllabic vocoid or `glide'
represented by the character \eg{y}) were kindly provided by Grzegorz
Kondrak.

As well as spoken similarity, the possibility of confusing antibody
names when written is also important, particularly in hand-written
notes or in prescriptions. Simpson\cite{simpson:visualsimilarity}
provides a letter confusion (or similarity matrix) which was
downloaded as an Excel file from the paper's supplementary material.

\begin{table}
\begin{center}
\begin{tabular}{llllllllllllll}\hline
          & a & b & c & d & e & f & g & h & i & j & k & l & m \\ \hline
Ascender  & 0 & 2 & 0 & 2 & 0 & 2 & 0 & 2 & 1 & 1 & 2 & 2 & 0 \\
Descender & 0 & 0 & 0 & 0 & 0 & 1 & 2 & 0 & 0 & 2 & 0 & 0 & 0 \\ \hline
          &   &   &   &   &   &   &   &   &   &   &   &   &   \\
          & n & o & p & q & r & s & t & u & v & w & x & y & z \\ \hline
Ascender  & 0 & 0 & 0 & 0 & 0 & 0 & 1 & 0 & 0 & 0 & 0 & 0 & 0 \\
Descender & 0 & 0 & 2 & 2 & 0 & 0 & 0 & 0 & 0 & 0 & 0 & 2 & 0 \\ \hline
\end{tabular}
\end{center}
  \caption{\label{tab:boumascores} Scores assigned for ascenders and
    descenders of each of the letters. All descenders were given a
    score of two with the exception of the letter \eg{f} whose score
    of one represents the fact that it often appears in printed text
    without a descender.}
\end{table}


Saenger\cite{saenger:bouma} introduced the term `Bouma shape'
(generally shortened to `Bouma') in typography to refer to the shape
of a cluster of letters, or more often to the shape of a whole
word. This was based on earlier work on the shapes of letters by
vision researcher Herman Bouma\cite{bouma:isolated,bouma:initial}.
The `Bouma' of a word written all in capitals is always a rectangle,
but lower-case words have ascenders and descenders protruding from the
basic rectangle. The importance of word shape in learning to read and
in word recognition has been discussed since the 1980s\cite[for
  example]{haber:wordshape}. Although some reading psychologists
question its importance (instead proposing the importance of `parallel
letterwise
recognition'\cite{rayner:span,paap:wordshape,larson:wordrecognition})
others still regard the Bouma as important\cite{glezer:bouma}.  The
program \verb|MakeBoumaMatrix| was written in Perl to generate a Bouma
similarity matrix based on the equation:
\begin{equation}
  B_{x,y} = M - (|A_x - A_y| + |D_x - D_y|)
\end{equation}
\begin{equation}
  M={\mathrm max}\{D_i, A_i\}_{(i\in \{a\ldots z\})}
\end{equation}
where $A_x$ is the ascender score for character $x$, $D_x$ is the
descender score for character $x$, $M$ is the maximal ascender or
descender score, and $B_{x,y}$ is the Bouma similarity for two letters
$x$ and $y$. Values for the ascender and descender scores are shown in
Table~\ref{tab:boumascores}.

The Needleman and Wunsch dynamic programming sequence alignment code
available in the BiopLib\cite{porter:bioplib} library was used in the
program \verb|abnamecheck| implemented in C. \verb|abnamecheck| allows
the analysis of all current names and allows a new name to be compared
against existing names.

As is common practice in computational linguistics, non-affine gap
penalties were used (i.e.\ the same penalty for introducing and
extending a gap). Initially gap penalties were set to equal the
maximum score seen in the respective scoring matrices and then reduced
until acceptable alignments were observed. For the Kondrak phonetic
alignments and Simpson visual similarity alignments, it was required
that the stem (\eg{-mab}) was correctly aligned and for the Bouma
alignments, it was required that the final letter \eg{b} was aligned
for at least 99.9\% of alignments.

From the analysis of existing names, the mean and standard deviation
of the similarities for each of these three scoring schemes was
calculated and a threshold of $\bar{x} + 3\sigma$ was used as a
default for identifying names that are significantly similar to names
already in the dataset.

The numbers of approved antibodies each year were obtained from the
`Approved Antbodies List' maintained by Janice Reichert and available
from the members area of the web site of the Antibody Society
(\url{www.antibodysociety.org}).


\begin{figure}
  \noindent a)
  \begin{center}
     \psfig{file=figures/namesperyear_cummulative.eps, width=0.75\linewidth}
  \end{center}
  \noindent b)
  \begin{center}
     \psfig{file=figures/ApprovedNamesPerYear_cummulative.eps, width=0.75\linewidth}
  \end{center}
  \caption{\label{fig:namesandapproved}Cummulative number of antibodies
  a)~named each year since 1991 when the \eg{-mab} stem was introduced and
  b)~first given marketing approval in Europe or the USA.}
\end{figure}



\begin{figure}
\begin{center}
  \psfig{file=figures/distribution.eps, width=0.7\linewidth}
\end{center}
\caption{\label{fig:distrib} Distribution of antibody name similarity
  scores from an all-by-all comparison of names in the complete
  dataset using 5\% bins. Similarity is scored using Kondrak phonetic
  similarity, Simpson visual letter similarity and the Bouma
  word-shape similarity.}
\end{figure}



\begin{figure}
  \noindent a)

\begin{center}
  \psfig{file=figures/mean_byyear.eps, width=0.7\linewidth}
\end{center}

  \noindent b)

\begin{center}
  \psfig{file=figures/sd_byyear.eps, width=0.7\linewidth}
\end{center}

  \noindent c)

\begin{center}
  \psfig{file=figures/meanplus3sd_byyear.eps, width=0.7\linewidth}
\end{center}

  
  \caption{\label{fig:results} Antibody name similariry
    scores plotted against year. a)~the mean score; b)~the standard
    deviation; c)~a threshold defined as
    $\bar{x}+3\sigma$. \note{Fix the legends to match Figure~2}}
\end{figure}



\section{Results}
The first antibody to be given an INN was muromonab in 1986 before the
\eg{-mab} stem was introduced; consequently this example was excluded
from the analysis. Antibodies which are conjugates (for example with a
toxin or with poly-ethylyne glycol) have a two-word name. For the
purposes of this analysis, the second word (naming the conjugate) was
removed so only the word ending in \eg{-mab} is compared.  Antibody
names extracted from the proposed and recommended lists are shown in
Supplementary File, {\tt names.csv} together with their INN
identifiers, list numbers and the year in which their names first
appeared in the proposed list. The cummulative number of antibodies
named and approved each year are shown in
Figure~\ref{fig:namesandapproved}. The accumulation of approved
antibodies follows approximately the same trend as those being given
names, but overall approximately 13\% of names antibodies have been
approved. The distribution of similarity scores for the full data set
is shown in Figure~\ref{fig:distrib}.

The mean similarity scores resulting from pairwise comparison of all
antibody names, were calculated for each year in a cummulative fashion
(for example the scores for 2010 were calculated for antibody names
publsihed in 2010 and in all previous years). Results are shown in
Figure~\ref{fig:results}. 

The earliest results are from 1991 when eight antibody drugs were
named: telimomab aritox (telimomab for the purposes of this analysis),
dorlimomab aritox (dorlimomab for this analysis), tuvirumab,
sevirumab, imciromab, maslimomab, nebacumab and biciromab. Given the
range of possible available names at that time, the diversity of
chosen names was small. In fact the average phonetic similarity score
of 62.98\% was the highest recorded as was the average
letter-confusion similarity score (60.88\%) and the Bouma
(85.29\%). However the standard deviations of the phonetic similarity
and letter-confusion scores were also high.

No more antibody names were allocated until 1995 where the diversity
of names increases considerably (indicated by a drop in the mean
similarity score) as it does in 1997. From that point on, the mean
similarity rises gradually until 2008 but falls again up until the
present time. Notably the standard deviations of the names follow a
somewhat different trend, showing a trend of decreasing since the late
1990s. 


\note{The results show XXX.}

\note{Do analysis within each year}

\note{Do analysis of each year against all pre-existing rather than
  all-vs-all.}

\note{What about looking at $\frac{SD}{\bar{x}}$ --- i.e. SD as a
  fraction of the mean.} 

\section{Discussion}
Over the last 10--15 years, antibody-based drugs have become a major
class of novel therapeutics with some 60--100 such drugs being named
each year. Only a small percentage of these are successful in gaining
marketing authorizations in Europe or the USA so many of these names
are `wasted'. The restrictions imposed by having a standard naming
scheme means that the name space available for assigning
nonproprietary names to these drugs is very restricted.

Surprisingly, we have shown that the name space has not become more
saturated over time, but that names are becoming more diverse.  The
changes made in 2017 (removing the source infix from the naming
scheme) allows an increase in diversity, but has had no significant
effect as yet. Nonetheless, we expect the rate at which antibody-based
drugs are developed to continue increasing. The change in naming
scheme in 2017 made no perceptable change to the trends in diversity
scores.

We have provided a tool that may help drug companies in suggesting
names which can be downloaded from \url{www.bioinf.org.uk/software/}
or from \note{\url{www.github.com/AndrewCRMartin/XXXX}}. The tool may also be
accessed for use over the web at \note{\url{www.bioinf.org.uk/abs/names/}}.


\bibliography{abbrev,checkname}
\end{document}


%% Supplementary Table S1 - antibody names, INN lists and years in
%% which they appeared



  

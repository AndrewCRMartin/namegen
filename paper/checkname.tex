\documentclass{article}
\usepackage{a4}
\usepackage{url}
\bibliographystyle{unsrt}

\newcommand{\eg}[1]{`\emph{#1}'}

\title{Non-proprietary name diversity for antibody-based drugs}
\author{Andrew C.R.\ Martin,\\Institute of Structural and Molecular
  Biology,\\Division of Biosciences,\\University College
  London,\\Darwin Building, Gower Street,\\London WC1E 6BT.}
\begin{document}
\maketitle

\begin{abstract}
Antibody-based drugs have become an extremely important class of
pharmaceuticals. Approximately one quarter of drugs assigned an
International Nonproprietary Name (INN) by the World Health
Organization (WHO) are antibodies.  The restrictions of the standard
INN naming scheme and the enormous number of names being assigned,
means that the available namespace is becoming densely populated.
Names that are too similar can lead to confusion and prescribing
errors. Here we investigate the diversity of the antibody name space
and show that the diversity has decreased over time. We provide a tool
that allows a potential new antibody name to be compared with
currently used names to help with selection of diverse names.
\end{abstract}

\section{Introduction}
Antibodies offer a number of advantages as drugs. They can act in the
same way that conventional drugs do (generally by binding to enzymes
or receptors and switching them on or off), but their large
interacting surface allows them to bind to targets (such as hormones,
growth factors and toxins) that small molecule drugs cannot bind
effectively since they do not have pockets into which small molecules
must bind. Secondly, antibodies can be used to trigger the iummune
system to kill cells or viruses. Thirdly, they can act to target other
drugs to the correct location in so-called `antibody-drug conjugates'
(ADCs). In addition, antibody-based drugs are being engineered to
allow cross-linking --- for example bringing cells of the immune
system into close proximity with cancer cells. Over the last 10--15
years, antibody-based drugs have become an incredibly important class
of pharmaceuticals with the majority of pharmaceutical companies and
many small biotechnology companies having antibody-based drugs in
their development pipelines\cite{xxx}. Approximately half of the top
ten selling blockbuster drugs are antibodiesc\cite{xxx} with
approximately one third of drugs in development being
antibodies\cite{xxx}.

The World Health Organization (WHO) assigns International
Nonproprietary Names (INN) to drugs. An INN allows a substance to be
identified uniquely and, when the substance comes off patent allows
other manufacturers to identify their products. The healthcare system
in many countries requires or encourages prescribing using
nonproprietary names since these are often considerably more cost
effective that original branded products. INNs are assigned to drugs
before they are licenced, typically during phase~II clinical
trials. The popularity of antibody-based drugs means that
approximately one quarter of all drugs that are being assigned an INN
are now antbodies.

The WHO uses a standard stem (\emph{-mab}) for almost all
antibody-based drugs, the exception being a small number of fusions of
antibodies with other biological drugs where the antibody is used only
to target the active substance to the correct site. Such molecules
have a \emph{-fusp} (`fusion protein') stem with the letter `\emph{a}'
appearing before the \emph{-fusp} to indicate that part of the fusion
is an antibody. Until recently, an infix before the \emph{-mab} was
used to name the source of the antibody. For example, \emph{-zu-} was
used to indicate the antibody had been humanized, \emph{-o-} for a
mouse antibody, \emph{-xi-} for a chimeric and \emph{-u-} for
human\cite{xxx} leading to antibody names frequently ending in
\emph{-zumab}, \emph{-umab}, or \emph{-ximab}. However, it became
clear that capturing the complexity of methods by which antibodies are
now produced in a single syllable was becoming impossible. This,
together with some controversies in the ways in which such
classifications were assigned, meant that these infixes were removed
from antibody names assigned from May 2017\cite{xxx}. Another infix is
also present which indicates the `target' for the antibody, although in
some cases this is more indicative of an initial indication for the
drug. The target substem includes \emph{-li-} for immunomodulatory,
\emph{-ci-} for circulatory system and \emph{-ta-} (previously
\emph{-tu-}) for tumours.

Thus, with the fixed stem (\emph{-mab}) and two infixes, the available
name space while trying to keep the name below five syllables was
severely limited and this was another reason for removing the source
infix. Even so, with the majority of antibodies targeting the immune
system or tumours, the vast majority of antibodies now being named end
in \emph{-limab} or \emph{-tamab} meaning that the diversity of
available names is still limited. Other restrictions on named include
the need to avoid relevant trademarks, or names that could be
associated with a company as well as avoiding names that could be
perceived to offer a commercial advantage or disadvantage (perhaps
because they are considered rude or offensive). Since these names are
used internationally, such considerations do not only apply in
English.

The enormous number of names being assigned means that the available
namespace is becoming densely populated. Names that are too similar
can lead to confusion and prescribing errors. Here we investigate the
diversity of the antibody name space using tools from computational
linguistics and show that the diversity has decreased over time. We
provide a tool that allows a potential new antibody name to be
compared with currently used names to help with selection of diverse
names.

\section{Materials and Methods}
Assigned antibody names were collected from the Mednet portal to the
INN data (\url{mednet-communities.net/inn}) covering all data from
proposed lists up to 118 and recommended lists up to 78 both released
in autumn 2017.

Comparison of names is essentially a problem of performing an optimal
alignment of two strings and is exactly the same problem faced in
sequence alignment. Standard dynamic programming algorithms for
local\cite{sw:alignment} or global\cite{nw:alignment} alignment are
employed to align two strings using appropriate scoring matrices. In
the case of protein sequence alignment, those scoring matrices are
generally either Dayhoff type matrices\cite{dayhoff:mdm} or BLOSUM
matrices\cite{henikoff:blosum}. In the case of linguistics, different
matrices may be chosen depending on the sorts of comparisons required:
one may wish to examine the similarity of words when written or when
spoken.

When examining the sounds of words, general computational linguistics
genetrally applies a pre-processing step known as grapheme-to-phoneme
(G2P) translation\cite{hixon:xxx}.  Phonemes are the perceptually
distinct units of sound that make up words. These are often written
using the International Phonetic Alphabet (IPA), a standardized
representation of the sounds of spoken
language\cite{macmahon:phoneticnotation} familiar to anyone who has
looked in a dictionary that contains pronunciation guidance. For
example, an obvious requirement would be to replace the letters
`\emph{ph}' and `\emph{f}' with the same phonetic character
(`{\bfseries\it f}') and this G2P translation is an essential step in
both automatic speech recognition and text to speech applications.
However, for the problem of comparing antibody names, this step is
somewhat superfluous. There is no `correct' pronunciation for these
words and dipthongs (either diagraphs representing a single sound such
as in the word `\emph{feat}' or compound vowel character ligatures
such as `\emph{\ae}') as well as double consonants, double vowels and
constructions such as `\emph{ph}' are not present. Thus for our
purposes, the names of antibodies can be compared directly with no G2P
translation.

Kondrak introduced a phonetic similarity matrix based around the
standard 26 letters of the
alphabet\cite{kondrak:paper,kondrak:thesis}. This methodology is
encoded in his ALINE program.  He classifies the base sound made by
each letter according to xxx features: xxxxxx (see Table~xxxx of
xxxx). Importantly, he also provides a set of saliancies (or weights)
for the features and weights that differ depending on whether the
comparison does or does not include a consonant. He provides a set of
equations that can be applied to these features to calculate the
similarity score for any pair of letters and we have implemented these
rules in the program \verb|MakeKondrakMatrix| written in C.  Corrected
features for the letter \eg{y} (or more strictly the non-syllabic
vocoid or `glide' represented by the character \eg{y}) were kindly
provided by XXXX Kondrak.

As well as spoken similarity, the possibility of confusing anotibody
names when written is also important, particularly in hand-written
notes or in prescriptions. Simpson\cite{simpson:xxx} provides a letter
confusion (or similarity matrix) which was downloaded as an Excel file
from the paper's supplementary material.

\begin{table}
\begin{center}
\begin{tabular}{llllllllllllll}\hline
          & a & b & c & d & e & f & g & h & i & j & k & l & m \\ \hline
Ascender  & 0 & 2 & 0 & 2 & 0 & 2 & 0 & 2 & 1 & 1 & 2 & 2 & 0 \\
Descender & 0 & 0 & 0 & 0 & 0 & 1 & 2 & 0 & 0 & 2 & 0 & 0 & 0 \\ \hline
          &   &   &   &   &   &   &   &   &   &   &   &   &   \\
          & n & o & p & q & r & s & t & u & v & w & x & y & z \\ \hline
Ascender  & 0 & 0 & 0 & 0 & 0 & 0 & 1 & 0 & 0 & 0 & 0 & 0 & 0 \\
Descender & 0 & 0 & 2 & 2 & 0 & 0 & 0 & 0 & 0 & 0 & 0 & 2 & 0 \\ \hline
\end{tabular}
\end{center}
  \caption{\label{tab:boumascores} Scores assigned for ascenders and
    descenders of each of the letters. All descenders were given a
    score of two with the exception of the letter \eg{f} whose score
    of one represents the fact that it often appears in printed text
    without a descender.}
\end{table}

Bouma\cite{xxx} introduced the importance of the shapes of words in
reading text. The `Bouma' of a word written in all-capitals is always
a rectangle, but lower-case words have ascenders and descenders
protruding from the basic rectangle. The importance of word shape in
learning to read and in word recognition has been widely
discussed\cite{xxx}.  A program \verb|MakeBoumaMatrix| was written in
Perl to generate a Bouma similarity matrix based on the equation:
\begin{equation}
  B_{x,y} = M - (|A_x - A_y| + |D_x - D_y|)
\end{equation}
\begin{equation}
  M={\mathrm max}\{D_i, A_i\}_{(i\in \{a\ldots z\})}
\end{equation}
where $A_x$ is the ascender score for character $x$, $D_x$ is the
descender score for character $x$, $M$ is the maximal ascender or
descender score, and $B_{x,y}$ is the Bouma similarity for two letters
$x$ and $y$. Values for the ascender and descender scores are shown in
Table~\ref{tab:boumascores}.

The Needleman and Wunsch dynamic programming sequence alignment code
available in the BiopLib\cite{porter:bioplib} library was used in the
program \verb|abnamecheck| implemented in C. \verb|abnamecheck| allows
the analysis of all current names and allows a new name to be compared
against existing names.

From the analysis of existing names, the mean and standard deviation
of the similarities for each of these three scoring schemes was
calculated and a threshold of $\bar{x} + 3\sigma$ was used as a
default for identifying names that are significantly similar to names
already in the dataset.

\section{Results}

\section{Discussion}

\bibliography{abbrev,checkname}
\end{document}


  
